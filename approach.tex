\section{Approaches}

To achieve broader usage it is important to require few if any changes to either the VR toolkit or the third party scientific visualization software, and to work as close as possible to the standard application development workflow. For VTK, this was accomplished by adding new features that fit within its the existing architecture.  The well-defined classes and API of VTK enabled us to primarily build upon existing code. VTK provides a well-defined rendering pipeline primarily consists of \texttt{RenderWindow}, \texttt{Renderer}, \texttt{Camera}, \texttt{Actor}, and \texttt{Mapper} classes. In the next section we will cover details on these components from architecture point of view. In the implementation sub-section, we will provide details on in depth detail on features we implemented to support configurable immersive scientific visualization applications. 

\subsection{OpenGL context sharing}

Traditionally, VTK creates and manages its own OpenGL context and the data objects within the scene. The objective of this work is to bring the high-quality scientific visualization computing and rendering capabilities of VTK to virtual reality environments in a way that is easier to develop and maintain. By bringing VTK into virtual environments created by domain-specific tools such as GLUT, VRUI, and FreeVR, we are providing the tools necessary to build interactive, 3D scientific visualizations to the developers of the virtual reality community.

\subsubsection{Architecture}

Integrating VTK in external rendering systems required overriding some of the behavior of the \texttt{vtkRenderWindow}, \texttt{vtkRenderer}, and \texttt{vtkCamera} classes. A \texttt{Renderer} is attached to a \texttt{RenderWindow}, a \texttt{Mapper} to an \texttt{Actor}, and a \texttt{Camera} to a \texttt{Renderer}. In a typical VTK application the \texttt{RenderWindow} class is responsible for creating a rendering context, and defining width and height of the visualization viewport. The \texttt{Renderer} class is responsible for rendering one-or-more \texttt{Actor}s. The \texttt{Actor} class is a drawable entity, which uses a \texttt{Mapper} to render specific data within a \texttt{Renderer}. Each of these components has their corresponding derived classes that implements the API using OpenGL, VTK's underlying graphics API. Using OpenGL provides VTK with the ability to use hardware acceleration that ultimately leads to better visualizations and near real-time performance as required by many interactive applications including the ones that are designed for immersive environments. In this design of VTK, each component participates in a specific way and communicates with other components via the public API provided by each. For instance, the \texttt{RenderWindow} typically creates the context in which \texttt{Renderer} draws drawable entities, \texttt{Actor}s. A \texttt{RenderWindow} can have one or more \texttt{Renderer}s. Each \texttt{Renderer} can make a decision on whether or not it should reset the buffers such as color or depth to its initial state while rendering one-or-more actors. 

Since \texttt{vtkRenderWindow} typically creates the context, and \texttt{vtkRenderer} controls objects of a scene in a given viewport, the rendering pipeline is constructed with properties and other attributes set specifically to support this most general use case. However, in the case of external environments, the context is created outside of VTK, and non-VTK graphical elements (such as the GUI) may be rendered before or after the VTK rendering. In addition, the environment may render its own visualization objects in the same context. To handle this situation, we have introduced a new module in VTK called \texttt{vtkRenderingExternal} that comprises four new classes: \texttt{vtkExternalOpenGLRenderWindow}, \texttt{vtkExternalOpenGLRenderer}, \texttt{vtkExternalOpenGLCamera} and \texttt{ExternalVTKWidget}.

The \texttt{vtkExternalOpenGLRenderWindow} class is an extension to the \texttt{vtkGenericOpenGLRenderWindow}, which provides a platform-agnostic VTK OpenGL window. The external render window class prevents a new VTK render window from being created and, instead, uses an existing OpenGL context. The \texttt{vtkExternalOpenGLRenderer} derives from \texttt{vtkOpenGLRenderer} and provides all of its features and functionalities. The external renderer offers an API that prevents it from clearing the OpenGL color and depth buffers at each frame. This ensures that the main application holds control over the OpenGL context and preserves rendered elements in the scene, of which VTK is unaware.

\subsubsection{Implementation}

One of the most important prerequisites of this work was seamless stereo rendering and user interaction with the two rendering systems. 

\textbf{\textit{Stereo Rendering}} The OpenGL context maintains the state machine in which OpenGL commands change the state of the system or query a particular state as needed. To support stereo, we utilized the OpenGL context, set by the VR toolkit, to determine the the type of stereo (Quad Buffer, Side-by Side, or Top-Bottom stereo) and simply render using the OpenGL context, which sets active buffer, stenciling, etc. This is set only once immediately after the context has been created and maintained by the VR toolkit over time. 

\textbf{\textit{2D and 3D Interface Widgets}} It is unlikely that the VTK elements will be the only rendering in a scene. There will probably at least be some GUI elements that will also be rendered. Thus, the VTK rendering will be mixed with other OpenGL elements. The new \texttt{ExternalRenderer} class does not clear the depth or color buffers, leaving that to the display integration library or application. The depth buffer can then act to allow OpenGL elements to be mixed (composited) in three-space with closer elements occluding farther ones.

\textbf{\textit{User Interactions}} Generally, in the case of a VR toolkit, interaction such as navigation in the scene space, grab, and rotation of various scene objects are handled by the system (e.g., Vrui). VTK has its own classes and methods for interaction and scene object manipulation. To synchronize the navigation in these two systems, the \texttt{vtkExternalOpenGLCamera} class was added. This class empowers the external application to manage camera interaction for VTK objects. We added a GL query in the external renderer, which uses the GL state system to get the projection and modelview transformation matrices. These two matrices determine the location and orientation of the object in the scene. The \texttt{vtkExternalOpenGLRenderer} sets these matrices on the \texttt{vtkExternalOpenGLCamera}. Setting these matrices directly on the camera leaves the camera parameters such as position, focal point, and view up direction to incorrect values, Therefore, we compute appropriate viewing coordinates for the camera by multiplying the modelview martix with the camera initial default position that is in OpenGL coordinate system.  Once, everything is set on the camera, the navigation and lighting works as expected by the user.

The application itself handles the secondary kind of interactions such as interactive slicing and clipping of the scientific datasets. VTK provides classes (filters) to perform thresholding, clipping, slicing, etc. These filters take inputs such as thresholding value, slicing position and normal, and clip position.  In our implementation, the application receives the tracker data (6 DOF), and, based on the mode the application is in, uses these information to set appropriate values on a specific filter. This integration is straightforward as our module makes the coordinate system consistent between the two rendering system.

\subsubsection{Enabling \texttt{vtkRenderingExternal}}

The \texttt{ExternalVTKWidget} class provides a one-stop solution to use all the new classes, described above, in an external application. The overarching application need just instantiate this class to use VTK's rendering capabilities. It creates a new external render window or uses one provided to it.

This work has been merged into the VTK as of release 7.0 available at www.vtk.org. To enable this module, set Module \texttt{vtkRenderingExternal} to \texttt{ON} (default is \texttt{OFF}).

\subsection{VR Toolkit embedding (OpenVR)}

To make it possible to use OpenVR-compatible devices with VTK, we embedded OpenVR into VTK within a module, called \texttt{vtkOpenVR}. Our goal is to allow VTK programs to use the OpenVR library with few changes, if any. If you link your executable to the \texttt{vtkOpenVR} module, the object factory mechanism should replace the core rendering classes (e.g., \texttt{vtkRenderWindow} and \texttt{vtkRender}) with the OpenVR-specialized versions in VTK. 

\subsubsection{Implementation}

The \texttt{vtkOpenVR} module contains the following classes as drop-in replacements in VTK.

\textbf{\texttt{vtkOpenVRRenderWindow}} - This is a derived classes of the RenderWindow class. The current implementation creates one renderer that covers the entire window. As described in the Related work section, this class (and \texttt{vtkOpenVRRenderer}) is the location for embedding the VR toolkit, and handles the bulk of interfacing to OpenVR. 

\textbf{\texttt{vtkOpenVRRenderer}} - This is a derived classes of the Render class. The \texttt{vtkOpenVRRenderer} class computes a reasonable scale and translation, and sets the results on \texttt{OpenVRCamera}. It also sets an appropriate default clipping range expansion. Again, this class (and \texttt{vtkOpenVRRenderWindow}) is the location for embedding the VR toolkit.

\textbf{\texttt{vtkOpenVRCamera}} - This is a derived classes of the Camera class.\texttt{vtkOpenVRCamera} gets the matrices from OpenVR to use for rendering. It contains a scale and translation that are designed to map world coordinates into the head-mounted display (HMD) space. Accordingly, the application developer can keep world coordinates in the units that are best suited to his/her problem domain, and the camera will shift and scale the coordinates into the units that make sense for the HMD.

\textbf{\texttt{vtkOpenVRRenderWindowIneractor}} - VTK is designed to pick and interact based on two-degrees of freedom, desktop X and Y mouse/window coordinates. In contrast, OpenVR provides X, Y and Z world coordinates and W, X, Y and Z orientations. The \texttt{vtkOpenVRRenderWindowInteractor} class catches controller events and converts them to mouse/window events. In addition, this class also stores the world coordinate positions and orientations for the styles or pickers that can use them. \texttt{vtkOpenVRRenderWindowInteractor} supports multiple controllers through the standard PointerIndex approach that VTK uses for MultiTouch.

\textbf{\texttt{vtkInteractorStyleOpenVR}} - In concert with the \texttt{vtkOpenVRRenderWindowInteractor} class, we derived the \texttt{vtkInteractorStyleOpenVR} class. The \texttt{vtkInteractorStyleOpenVR} class uses X, Y and Z world coordinate positions and W, X, Y and Z orientations to adjust \texttt{Actor}s. This class provides a nice grab-and-move style of interaction that is common to OpenVR and other VR toolkits.

\textbf{\texttt{vtkOpenVRPropPicker}} - Finally, the derived \texttt{vtkOpenVRPropPicker} class determines what \texttt{Actor}s or \texttt{Prop}s VTK picks. Note that \texttt{Prop} is an abstract superclass for any objects that can exist in a rendered scene (either 2D or 3D), and defines the API for picking, LOD manipulation, and common instance variables that control visibility, picking, and dragging. The \texttt{vtkOpenVRPropPicker} class uses the X, Y and Z world coordinate as the picking value as opposed to an intersecting a ray, which is slower.

These OpenVR derived classes work from within VTK to provide the seamless access cameras, lighting, interaction and the complete VTK pipeline.

\subsubsection{Enabling \texttt{vtkOpenVR}}

To use \texttt{vtkOpenVR}, first download the master branch of VTK from the VTK repository on GitHub (see www.vtk.org). The remote module for \texttt{vtkOpenVR} can be found at https://goo.gl/0jem0V. Place this file into the Remote folder of your VTK source tree. \texttt{vtkOpenVR} also requires that you download two external libraries: Simple DirectMedia Layer 2 (SDL2) and OpenVR. To enable this module, set Module \texttt{vtkOpenVR} to \texttt{ON} (default is \texttt{OFF}).Make sure you build an optimized version of VTK to maximize performance.

\subsubsection{Future Developments}

The \texttt{vtkOpenVR}  module is currently in the alpha phase and currently tested on the HTC Vive virtual reality system. Moving forward, we look to add support for the OpenVR overlay, which is great for displaying a user interface. We also aim to make the module faster and include more event interactions. 

\subsection{Performance enhancements}

The Visualization Toolkit (VTK) is one of the most commonly used libraries for visualization and computing in the scientific community. Primarily written in C++, VTK provides classical and model visualization algorithms to visualize structured, unstructured, and point data sets on desktop, mobile, and web environments. The open source, community driven VTK provides state-of-the-art implementations that are only an API call away. The benefit in using VTK comes from the fact that having the latest algorithm implementation simply requires using the existing implementation in VTK or contributing the it to VTK.

\subsubsection{OpenGL 3.2+}

The legacy rendering code in VTK is a group of implementation modules collectively called ``OpenGL." Through a grant from the National Institutes of Health, the OpenGL group has been rewritten as a drop-in replacement set of implementation modules collectively called``OpenGL2.? This work aims to support rendering on modern graphics cards~\cite{Hanwell:2015}.

The results have been nothing short of spectacular. For polygon rendering demonstrates a ten times speedup for first frame rendering followed by a two-hundred times speed up for subsequent frames for one to thirty million triangles. The previous volume rendering was also graphics processing unit (GPU) aware, and, thus, the improvement is a modest but substantial two times speedup. 

To realize these performance enhancements, VTK depends on an OpenGL 3.2+ context, which is available on fairly low end modern GPUs. However, for those application developers using the X11 window system on a Mac OSX system, xQuartz does not provide a suitable OpenGL context currently. But, as xQuatrz utilizes newer versions of Mesa going forward, we expect future versions will eventually fill the OpenGL2 requirements.

\subsubsection{Dual-Depth Peeling}

As we developed several example programs leveraging the \texttt{vtkRenderingExternal} module, we found that the rendering performance slowed as transparency was introduced into the scene. 

In OpenGL, polygons are broken up into fragments through the rasterization process. Each fragment corresponds to a pixel. An OpenGL fragment shader is a customizable program that determines the color of a fragment where all fragments for a single pixel are blended by OpenGL to determine the final color of the pixel. Composing multiple translucent fragments into a single pixel must be done carefully. There are three common strategies to this composition:

\begin{compactitem}
\item \textbf{Simple Alpha Blending} - The fragments are processed (blended using just alpha) in random order. It is very fast, but provides unpredictable and generally incorrect results.
\item \textbf{Sorted Geometry} - Geometry must be resorted each time the camera moves using \texttt{vtkDepthSortPolyData}. Sorting is an expensive (slow) operation, but provides generally consistent results with some artifacts where primitives overlap.
\item \textbf{Depth Peeling} - Extract and blend fragments in a multipass render, and, therefore, requires multiple geometry render passes.
\end{compactitem}

VTK by default uses depth peeling. To enhance rendering performance with transparancy we implemented \texttt{vtkDualDepthPeelingPass}, which was originally proposed by nVidia in 2008~\cite{Bavoil:2008}. Dual-depth peeling extends traditional depth peeling by extracting two layers of fragments per-pass: from the front and back simultaneously. Uses a two-component depth buffer to track of peel information and three types of geometry passes:

\begin{compactitem}
\item \textbf{InitializeDepth} - Initializes buffers using opaque geometry information.
\item \textbf{Peeling} - Repeated pass that extracts and blends translucent geometry peels. It extracts both near and far peels while blending far peels into accumulation buffer.
\item \textbf{AlphaBlending} - An optional pass to clean up unpeeled fragments and used with occlusion thresholds.
\end{compactitem}

This work provides a two times speedup for compositing in the appearance of transparent geometry.

\subsubsection{vtkSMPTools}

The field of parallel computing is advancing rapidly due to innovations in GPU and multicore technologies. The VTK community is working to make parallel computing for scientific visualization easier by introducing \texttt{vtkSMPTools}, an abstraction for threaded processing which under the hood uses different libraries such as TBB, OpenMP and X-Kaapi. The typical target application is coarse-grained shared-memory computing as provided by mainstream multicore, threaded CPUs such as Intel's i5 and i7 architectures.

For several of the example programs utilizing the \texttt{vtkRenderingExternal} module, we leveraged a new contouring algorithm in VTK that is readily parallelizable using \texttt{vtkSMPTools} and still incredibly efficient in serial mode, \texttt{vtkFlyingEdges2D} and \texttt{vtkFlyingEdges3D}. While the OpenGL2 group improves rendering performance, \texttt{vtkSMPTools} can be used to enhance the geometry generation performance for scientific visualization.