\abstract{Modern scientific, engineering and medical computational simulations, as well as experimental and observational data sensing/measuring devices, produce enormous amounts of data.
While statistical analysis provides insight into this data, scientific visualization is tactically important for scientific discovery, product design and data analysis.
These benefits are impeded, however, when scientific visualization algorithms are implemented from scratch|a time-consuming and redundant process in immersive application development.
This process can greatly benefit from leveraging the state-of-the-art
open-source Visualization Toolkit (VTK) and its community.
Over the past two (almost three) decades, integrating VTK with a virtual reality (VR) environment has only been attempted to varying degrees of success.
In this paper, we demonstrate two new approaches to simplify this amalgamation
of an immersive interface with visualization rendering from VTK.
In addition, we cover several enhancements to VTK that provide near real-time updates and efficient interaction.
Finally, we demonstrate the combination of VTK with both Vrui and OpenVR immersive environments in example applications.
} % end of abstract

\keywords{Scientific visualization, immersive environments, virtual reality}

%% ACM Computing Classification System (CCS). 
%% See <http://www.acm.org/class/1998/> for details.
%% The ``\CCScat'' command takes four arguments.

\CCScatlist{ % not used in journal version
 \CCScat{I.3.6}{Computer Graphics}{Methodology and Techniques}{Interaction Techniques};
 \CCScat{I.3.7}{Computer Graphics}{Three-Dimensional Graphics and Realism}{Virtual Reality};
 \CCScat{H.5.2}{Information Interfaces and Representation}{User Interfaces}{Interaction Styles}
}
