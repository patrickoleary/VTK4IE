\textbf{\textit{Bill, the following should end the introduction and spells out our contributions.}}

Our two approaches for the amalgamation of VTK and an immersive environment through a VR Toolkit, makes several contributions to immersive scientific visualization.

\textbf{OpenGL context sharing}. Our vtkRenderingExternal VTK module provides a complete integration including lights, interaction, picking and access to the entire VTK pipeline. This, in turn, enables simple utilization for application developers with any OpenGL-based VR Toolkit.

\textbf{VR Toolkit embedding}. The OpenVR VTK module supports several immersive environments now without the issues faced by previous work, and is a complete template for embedding other VR Toolkits within VTK in the future.

\textbf{Enhanced performance}. enhancements to VTK that significantly impact immersive environment application development. These enhancements include:

\begin{compactitem}
\item The new default OpenGL 3.2+ pipeline;
\item dual depth peeling for transparency; and 
\item symmetric multiprocessing (SMP) tools and algorithms.
\end{compactitem}

Finally, we have exposed the framework of our image-based approach to the scientist through an advanced selection interface that allows them to make sophisticated (time, storage, analysis, ...) decisions for the production of \textit{in situ} visualization and analysis output.

In the sections that follow, we illustrate how our amalgamation of VTK and VR Toolkits supports our goals for enhance scientific visualization in immersive environments.